\documentclass{ctexart}
\usepackage[a4paper, margin=1in]{geometry}
\usepackage{graphicx}
\usepackage{caption}
\usepackage{subcaption}
\usepackage{enumitem}
\usepackage{setspace}
\usepackage{amsmath}
\usepackage{xcolor} 
\usepackage{hyperref}

\hypersetup{
    colorlinks=true,
    urlcolor=blue,
}



\newcommand{\subtitletext}[1]{{\fontsize{16}{16} \textbf{#1} \vspace*{1ex}}}

\newcommand{\codebox}[1]{\colorbox[rgb]{0.9, 0.9, 0.9}{\texttt{\raisebox{0pt}[8pt][0pt]{#1}}}}

\begin{document}

\setstretch{1.5}
\setlength{\parskip}{1ex}
\setlength{\parindent}{0pt}
\renewcommand{\labelenumii}{(\arabic{enumii})}

\fontsize{12}{12}

\subtitletext{问题背景}

“算24点”是一种数学游戏, 规则如下:

任意抽取 4 张扑克牌 (不含大、小王), 记下抽取的牌面上的数字, 
其中 \codebox{A}、\codebox{J}、\codebox{Q}、\codebox{K} 分别视为 \codebox{1}、\codebox{11}、\codebox{12}、\codebox{13};
目标是找到一个由这 4 个数字组成的四则运算式, 4 个数字均使用且仅使用一次, 且算式的结果等于 \codebox{24}.

\vspace*{4ex}

\subtitletext{问题描述}

程序的核心任务是实现一个 “算24点” 的计算器, 读取输入的 4 个牌面数字, 判断是否存在符合条件的算式;
如果有解, 则给出一个算式.

此外, 项目还需要实现下列额外功能:

\begin{itemize}[label=*, itemindent=0pt, leftmargin=12pt]
    \item 文件处理功能. 程序需要读取指定文件, 文件的每一行包含一组输入; 
    程序分别判断对于每组输入, 是否存在符合条件的算式;
    完成后, 程序可以导出一个输出文件, 内容包含每组输入和判断结果.

    \item 计时挑战功能. 由程序生成“算24点”题目, 用户挑战在规定时间内做出尽可能多的题目.
    具体地说, 程序设置一个倒计时, 每次生成一组 4 个数字作为题目, 保证题目有解;
    用户需要给出一个算式, 由程序判断它是否是符合要求的答案.
    若答案正确, 用户会获得分数奖励; 若答错或放弃题目, 会相应受到扣除分数惩罚;
    用户需要挑战在倒计时结束前获取高分.

    \item 联机对战功能. 程序可以在局域网内的两台计算机下建立连接, 两位用户分别控制一台计算机;
    由程序生成“算24点”题目, 用户中最先得出正确结果的一方获得加分, 先达到指定分数的一方获胜.
    由于网络相关功能有一定难度, 该功能目前暂未完成开发.
    
\end{itemize}

项目使用的算法、界面设计等详见下文.

\vspace*{4ex}

\subtitletext{测试用例}

输入:$\codebox{[3, 3, 8, 8]}$

输出一组符合要求的解: $\codebox{8 / (3 - 8 / 3) = 24}$

输入:$\codebox{[6, 7, 8, 8]}$

没有符合要求的解, 输出: $\codebox{无解}$

\newpage

\subtitletext{算法设计}

程序需要实现的主要算法功能包括:

\begin{itemize}[label=*, itemindent=0pt, leftmargin=12pt]
    \item 求解“算24点”问题, 即对于给定的 4 个数字, 通过搜索回溯法, 枚举可能的运算符和运算顺序, 找到一组可行的解;
    
    \item 输出答案表达式, 即求解得到结果时, 将当前的搜索路径保存为表达式树, 并转换为中缀表达式输出;
    
    \item 表达式求值, 即处理用户挑战时输入的答案, 将中缀表达式转换为表达式树并求值, 判断答案是否满足要求.
    
\end{itemize}

下面列出关键的实现: 

\begin{enumerate}[itemindent=0pt, leftmargin=14pt]
    \item 有理数类 \codebox{rational} :
    
    考虑到浮点类型的精度问题, 定义有理数类进行程序中的四则运算. 有理数类重载了四则运算和比较运算符.

    \item 表达式树模板类 \codebox{expression<T>} :
    
    每个对象保存一个表达式树的结点. 结点分为操作数结点和运算符结点, 
    操作数结点存储一个 \codebox{T} 类型的操作数对象;
    运算符结点存储 2 个 \codebox{expression<T> *}, 即左右子表达式树的指针,
    并以枚举类型 \codebox{enum operator\_type} 的形式存储一个运算符,

    表达式树类实现了以下方法:

    \begin{enumerate}[itemindent=0pt, leftmargin=16pt]
        \item \codebox{T value()}, 返回表达式的值:
        
        对于操作数结点, 返回操作数; 对于运算符结点, 递归地求左、右子表达式的值, 根据运算符种类返回对应运算的结果.
        
        \item \codebox{void print\_expression\_to(std::ostream \&)}, 打印为中缀表达式:
        
        对表达式树进行中序遍历, 输出对应的操作数和运算符. 
        若子树的运算符的优先级较低, 应在子树的表达式两侧加括号;
        若右子树的运算符的优先级相同, 但父结点的运算符为 \codebox{-} 或 \codebox{÷} 时,
        打印右子树时应反转运算符.
        
        \item 构造函数 \codebox{expression(const std::string \&)}, 由中缀表达式建树:

        用两个栈分别保存表达式和运算符, 并维护运算符栈的单调性质.
        当从字符串中读取到操作数时, 构造一个单操作数结点的表达式树, 压入表达式栈中;
        当读取到运算符, 且栈顶的运算符结合优先级更高时, 栈顶的运算符出栈;
        运算符出栈时, 表达式栈顶的两个表达式出栈, 它们运算得到的表达式入栈.
        \codebox{(}直接进入运算符栈, 当读取到\codebox{)}时, 栈中运算符依次出栈, 直到与\codebox{(}配对.

    \end{enumerate}

    \newpage

    \item 求解器类 \codebox{solver} :
    
    用于实现“算24点”及类似问题的求解, 以及处理求解过程中产生的临时数据.

    初始化求解器时, 需要确定操作数的数量及目标答案, 对于“算24点”问题分别是 \codebox{4} 和 \codebox{24}.

    求解器通过搜索回溯求解; 若当前有 \codebox{N} 个操作数, 程序枚举四则运算符, 
    并枚举从中选出 \codebox{2} 个数的所有组合 (或排列, 取决于当前运算符是否满足交换律);
    对于每一种枚举情形, 将这 \codebox{2} 个数取出, 运算后放回, 得到 \codebox{N-1} 个操作数,
    并调用下一层搜索, 以此类推. 最后剩下一个数, 若这个数恰好是目标答案, 判断问题有解, 此时保存对应的表达式; 若不是目标答案, 则进行回溯.
    若完成了所有搜索, 没有找到目标答案, 则判定问题无解.

    文件处理允许在设置中启用多线程并发操作, 此时每个线程会创建单独的求解器实例, 确保数据互不干扰.
    
    
\end{enumerate}

对于其余辅助性的实现, 可以在代码中查阅, 此处不再赘述.

\vspace*{4ex}

\subtitletext{界面设计}

项目使用 \codebox{Qt6} 实现 GUI, 主要的界面包括:

\begin{enumerate}[itemindent=0pt, leftmargin=14pt]

    \item 主窗口:
    
    左侧为菜单栏, 选中对应的菜单选项时, 右侧显示对应的界面.

    \item 计算器界面:
    
    中部提供 4 个编辑栏, 允许用户输入“算24点”问题的 4 个参数, 并展示对应的扑克牌面;

    下方提供随机取数按钮、求解按钮和一个用于显示结果的只读编辑栏.

    \item 文件处理界面:
    
    上方提供一个编辑栏, 用于输入待求解文件的路径; 右侧为通过浏览选取文件的按钮, 所选文件的路径也会显示在左侧编辑栏中;

    下方提供执行计算按钮、一行状态文本和保存结果按钮.

    \item 计时挑战界面:
    
    上方为计分板, 左侧显示倒计时, 右侧为当前分数与最高记录;

    中部展示 4 个只读的编辑栏和扑克牌面, 用于展示当前题目;

    下方在游戏开始前为开始按钮;
    游戏开始后, 显示输入答案的编辑栏、刷新题目按钮、提交按钮 (在编辑栏中键入回车也可以提交);
    提交答案后, 编辑栏上方的文本和右侧的图标会显示答案是否正确, 以及用户的得分变动;
    游戏时间结束后, 显示开始按钮, 以及上一局的分数.

    \item 设置界面:
    
    提供各功能所需的相关选项等.
    
\end{enumerate}

\subtitletext{附注}

项目采用 \codebox{git} 进行版本管理, 开发进度同步到 \href{https://github.com/lastrivia/Coursework_Calc24}{GitHub} 上;
您可以查询 \codebox{commmit} 历史记录以了解项目的开发经过; 

项目使用的图标、插画等美术资源使用 \codebox{Adobe Illustrator} 绘制, 部分参考了网络上的图片资源;

作业中算法模块的设计与实现, 由学生独立完成; GUI 模块的开发, 部分借助了 AI 工具以检索 \codebox{Qt} 的技术文档和相关资料.

\end{document}